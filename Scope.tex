\section{Scope}
% Add a section and label it so that we can reference it later

This guide prescribes the format and contents of the concept of operations (ConOps) document. A ConOps is a user-
oriented document that describes system characteristics of the to-be-delivered system from the user’s viewpoint. The
ConOps document is used to communicate overall quantitative and qualitative system characteristics to the user, buyer,
developer, and other organizational elements (e.g., training, facilities, staffing, and maintenance). It describes the user
organization(s), mission(s), and organizational objectives from an integrated systems point of view.
This guide may be applied to all types of software-intensive systems: software-only or software/hardware/people
systems. The concepts embodied in this guide could also be used for hardware-only systems, but this mode of use is
not addressed herein. The size, scope, complexity, or criticality of the software product does not restrict use of this
guide. This guide is applicable to systems that will be implemented in all forms of product media, including firmware,
embedded systems code, programmable logic arrays, and software-in-silicon. This guide can be applied to any and all
segments of a system life cycle.
This guide identifies the minimal set of elements that should appear in all ConOps documents. However, users of this
guide may incorporate other elements by appending additional clauses or subclauses to their ConOps documents. In
any case, the numbering scheme of the required clauses and subclauses should adhere to the format specified in this
guide. Various clauses and subclauses of a ConOps document may be included by direct incorporation or by reference
to other supporting documents

