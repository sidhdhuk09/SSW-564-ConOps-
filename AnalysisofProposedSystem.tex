\chapter{Analysis of the proposed system \\
%\small{\textit{-- Team-11}} 
\index{Chapter!Analysis of the proposed system }
\index{Analysis of the proposed system}
\label{Chapter::Analysis of the proposed system}}\

The system "Clauseguard" provides several benefits which are highlighted below: 
\begin{itemize}
    \item The "ClauseGuard" system mundanes access to legal advice, bringing complex knowledge of terms and conditions contracts to the fingertips of the everyday user. By leveraging advanced machine learning algorithms, it empowers individuals and organizations with the ability to identify potentially fraudulent clauses without the need for extensive legal consultation.

" 
    \item By automating the examination process, "ClauseGuard" drastically reduces the margin of human error that may occur even with the involvement of experienced legal consultants. The model's capacity to scan vast amounts of data ensures a thorough, comprehensive review of contractual content, which traditionally could take considerable time and resources.
    \item The system is designed with user-friendliness at its core, making it accessible to any stakeholder with basic computational skills. This opens up the domain of legal understanding, traditionally seen as complex and out of reach for many, to a broader audience. It democratizes the process of understanding and evaluating legal contracts, thereby helping users make more informed decisions.
    \item "ClauseGuard" not only brings about significant cost savings but also enhances the speed and efficiency of contract reviews. It can be utilized by small businesses, corporations, and individual users, thereby promoting a culture of transparency and fairness in contractual agreements.
    \item The system's flexibility allows it to adapt to varying legal environments and contract structures. This means that as legal language and contract formats evolve, "ClauseGuard" can be updated and trained to understand these changes, ensuring it remains a valuable tool for contract review.

    \subsection{Limitations}
    Despite the advantages clauseguard brings to the everyday world, it also has certain limitations which are highlighted below: \begin{itemize}
        \item The performance of the model is contingent on the quality of the data it's trained on. Poor quality data can result in inaccurate detections.
        \item As legal language and fraud strategies evolve, the model will need continuous updates and training to maintain its effectiveness.
        \item As clauseguard is being deployed on an international scale, this presents with a unique set of challenges. Given the varying legal definitions and structures across different countries, the system must be equipped to incorporate a vast array of terminologies and documents. The complexity of the task is amplified by the intricate and distinctive nuances of legal language in different jurisdictions. The system's ability to effectively interpret and adapt to these differences is crucial in its functionality and effectiveness.


        \item  "ClauseGuard" must not only recognize nuances but also articulate them to the user in a comprehensible manner. Translating complex legal jargon into user-friendly language is no simple task, particularly when considering the diverse and intricate nature of legal contracts. This translation process, however, is essential to meet the system's goal of making legal knowledge accessible to the common man.
        \item It is also extremely important to note the potential ethical implications of an automated system providing legal advice. While clauseguard is designed to identify potentially deceptive clauses, it's not a replacement for professional legal advice. The model must be clear in communicating its role as a supportive tool rather than a definitive legal authority.
        




        
    \end{itemize}

\subsection{Advantages}
        Clauseguard provides several key advantages. 
        \begin{itemize}
            \item The system can easily scale to handle larger volumes of contracts without a proportional increase in resources or time.
            \item Unlike human reviewers, "ClauseGuard" offers consistent performance, unaffected by factors like fatigue or bias.


        \end{itemize}
        \subsection{Disadvantages}
        Clauseguard has some disadvantages. 
        \begin{itemize}
            \item Potential for False Positives/Negatives: The model may occasionally flag non-fraudulent clauses as fraudulent (false positives) or miss fraudulent clauses (false negatives).
            \item There can be significant initial costs associated with setting up the system and training the model.
            \item The efficiency of ClauseGuard is directly proportional to the quality of data it receives. If the input data is poorly written or ambiguous, the model's ability to accurately detect deceptive clauses could be compromised.

            \item  While machine learning models can analyze and draw conclusions from vast amounts of data, they lack the human capacity for intuition and contextual understanding. There might be cases where human judgment could provide more nuanced interpretations of contract clauses upheld in a court of law.
            \item The potential for misuse or over-reliance on ClauseGuard could lead to legal liability issues. If a user were to take action based on the model's suggestions and suffer negative consequences, it might raise questions about who is legally responsible.
            \item ClauseGuard may not be fully effective for highly specialized or unique contracts that deviate from standard terms and conditions. Its performance may be limited in such cases.
            \item If ClauseGuard stores or processes sensitive contract data or confidential information, it could raise privacy concerns. Safeguarding user data would be a major responsibility and potential challenge.
            \item The model needs to be continually updated to stay current with changing laws, regulations, and contract norms. This could require significant resources over time.













        \end{itemize}
    \subsection{Alternatives and Trade-Offs Considered}
    Alternatives considered include continuing with manual review of contracts or expert paralegal terms. However, these options do not offer the same level of scalability and consistency, cost effectiveness and availability as ClauseGuard. The main trade-off is the initial investment in time and resources to set up and train the model, balanced against the long-term benefits in efficiency, accuracy, and cost savings.





        
\end{itemize}


\section{Summary of improvements \label{Section::Summary of Imporvements }}
The summary of benefits of clauseguard is as follows: 
\begin{itemize}
    \item New Capabilities: ClauseGuard introduces several  features, with main primary objective being the automated detection of potentially deceptive clauses in terms and conditions contracts. Th is capability is not typically found in traditional contract review processes, which often rely on manual review by legal professionals. Moreover, ClauseGuard can analyze vast quantities of contract data in a fraction of the time it would take a human reviewer, thus increasing efficiency. It also provides legal advice in the fingertips of the ordinary user thereby providing an expert opinion which would otherwise be financially stressful for the common person. 

    \item Enhanced Capabilities: For organizations already employing some form of contract review, ClauseGuard enhances these capabilities by adding a layer of machine learning-based analysis. This technology can identify patterns and anomalies that might be overlooked by human reviewers, leading to a more thorough and robust contract review process. 
    \item Deleted Capabilities: As ClauseGuard"becomes more prevalent, some obsolete or less efficient methods of contract review may be phased out. For example, labor-intensive manual reviews of every clause in a contract may no longer be necessary, saving time and resources. Paralegal teams may be downsized, saving firms specializing in matters of legal representation millions of dollars in added costs. 
    \item Improved Performance: ClauseGuard offers significant performance. Its response time is significantly faster than manual contract review, which can expedite contract negotiations and other business processes. The model's machine learning capabilities also mean that it can continually learn and improve over time, potentially leading to better quality contract reviews. Furthermore, as it's a digital tool, it does not require physical storage space, reducing resource requirements. It can also handle large volumes of data, which might be impractical for human reviewers especially on a time constraint.




\end{itemize}


\section{Disadvantages and limitations \label{Section::Disadvantages and Limitations}}
While the ClauseGuard system brings several significant improvements to the process of understanding legal ramifications of contracts, there are also potential disadvantages and limitations to consider: 
\begin{itemize}
    \item The introduction of ClauseGuard may necessitate retraining for personnel who are accustomed to traditional contract review methods. This training could involve understanding how to operate the system, how to interpret its output, and how to troubleshoot any issues that may arise. The time and resources required for this retraining could be significant, especially in larger organizations. 
    Note: This is similar to retraining of many writers, and experts after the advent of "ChatGpt" in many workplaces. Writers and editors are being retrained as "Prompt Engineers". These engineers are being trained on utilizing OpenAI's ChatGPT to write stories, fix grammatical mistakes and editing issues without the need for manual review. It is also causing significant loss to jobs in the fields of journalism, content writing and publishing. We expect the outcome with the introduction of clauseguard.
    \item The adoption of ClauseGuard might also entail a change in workflow or workspace organization. For instance, legal teams may have to adjust their work processes to integrate the output of the system into their contract review and negotiation procedures.
    \item While ClauseGuard is designed to detect deceptive clauses, it may not encompass all the features desired by users. For instance, it might not fully understand and interpret highly nuanced or complex legal language that varies from one jurisdiction to another.
    \item The adoption of ClauseGuard may lead to a certain degree of dependence on the tool, which might degrade existing capabilities, especially if personnel become less engaged in manual review processes.








\end{itemize}



\section{Alternatives and trade-offs considered \label{Section::Alternatives and trade-offs considered}}
During the development of clauseguard, we considered several trade-offs in our approach. These are discussed below: 
\begin{itemize}
    \item An initial consideration was whether to employ a rule-based system, where predefined rules are used to detect deceptive clauses, or to utilize a Machine Learning approach, where the system learns to identify such clauses based on patterns in the data.

Trade-offs:  A rule-based system could potentially be more understandable and easier to implement whilst maintaining the nuances of complicated legal jargon,  it would lack the flexibility and adaptability of an ML model to cover the changing legal landscape. It would also require extensive manual input for  development and maintenance.

Decision Reached: We decided to proceed with a Machine Learning model due to its superior adaptability and ability to handle vast and complex contract data as well as designing it to ensure it covers the latest changes in the legal landscape set forth by governments. 


\item Another alternative considered was whether to use supervised or unsupervised learning methods. Supervised learning would involve training the model on a labeled dataset of contract clauses, where deceptive clauses are pre-identified. On the other hand, unsupervised learning would not require pre-labeled data and instead would look for patterns and anomalies in the data to detect potentially deceptive clauses.
Trade-offs: While supervised learning could lead to a more accurate model if high-quality labeled data is available, it requires considerable effort to prepare such data. Unsupervised learning, meanwhile, can handle unlabeled data, making it potentially more scalable, but it might not be as accurate in detecting deceptive clauses.
Decision Reached: A supervised learning approach was chosen because of its potential for higher accuracy and the availability of a sufficient amount of pre-labeled contract clause data.


\item The third alternative considered for ClauseGuard was the choice of going for a web-based platform or an application centric platform. A web-based platform would allow users to access ClauseGuard through web browsers on various devices, including desktop computers, laptops, tablets, and smartphones. It would provide a consistent user experience across different devices and operating systems. Whereas, A application centric would provide users with a dedicated application either on desktop or mobile devices leveraging the  capabilities of their respective operating system and offering offline functionality to those in regions where always online requirement is difficult to obtain.

Trade-Offs: While a application centric approach would be safer, more secure and easier to develop, a web-based platform will guarantee that any stakeholder can access ClauseGuard from any geographic location around the world. In the unlikely case that the model is banned in certain geographic locations, the stakeholders can get around to it by the use of A virtual private network (VPN) to access ClauseGuard.

Decision Reached: After careful consideration, it was decided that a web-based platform would be the preferred approach for developing the model. Stakeholder feedback  indicated a significant preference for accessing ClauseGuard through a web-based interface rather than dedicated applications. This approach aligns with the goal of providing a user-friendly and easily accessible platform.














\end{itemize}



