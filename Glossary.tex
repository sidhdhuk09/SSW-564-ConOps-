\chapter{Glossary 
%\small{\textit{-- Team-11}} 
\index{Chapter!Glossary }
\index{Glossary}
\label{Chapter::Glossary}}\
\begin{itemize}
    \item ClauseGuard: The name of the system being developed. 

    \item ConOps: Abbreviation for Concept Of Operations. 

    \item ML: Machine Learning algorithm that can intelligently perform tasks without explicit coding.
    \item preprocessing: The initial stage in data analysis where raw data is cleaned and transformed to a suitable format for further analysis or model training.



    \item scikit-learn: Python library used for developing the machine learning model. 

    \item NLTK: Natural language toolkit used for preprocessing textual data.  
    \item Keras: A neural network API, used to train the model on preprocessed datasets. 
    \item AWS(Amazon Web Services): A subsidary of amazon providing cloud infrastructre. 
    \item Jira: A proprietary issue tracking product developed by Atlassian that allows bug tracking and agile project management.
    \item GitHub: A web-based hosting service for version control. 

    \item API (Application Programming Interface): A set of protocols, routines, and tools for building software and applications. APIs specify how software components should interact and allow different software systems to communicate with each other.

    \item API (Application Programming Interface): A set of protocols, routines, and tools for building software and applications. APIs specify how software components should interact and allow different software systems to communicate with each other.

    \item ChatGPT: An advanced language model developed by OpenAI, capable of generating human-like text based on the input given to it.

    \item Coverity: A static code analysis tool that helps developers and security teams address security and quality defects early in the software development life cycle.

    \item Velero: An open source tool to safely backup and restore, perform disaster recovery, and migrate Kubernetes cluster resources and persistent volumes.

    \item PDF (Portable Document Format): A file format developed by Adobe in the 1990s to present documents, including text formatting and images, in a manner independent of application software, hardware, and operating systems.
    \item LaTeX: A high-quality typesetting system. It is the de facto standard for the communication and publication of scientific documents.
    \item RTF (Rich Text Format): A proprietary document file format with published specification developed by Microsoft Corporation for cross-platform document interchange with Microsoft products.
    \item GDPR (General Data Protection Regulation): A regulation in EU law on data protection and privacy in the European Union (EU) and the European Economic Area (EEA). It also addresses the transfer of personal data outside the EU and EEA areas.
    
\end{itemize}