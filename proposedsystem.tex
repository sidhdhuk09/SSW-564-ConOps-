\chapter{Concepts for the proposed system \\
%\small{\textit{-- Team-11}} 
\index{Chapter!Concepts for the proposed systems }
\index{Concepts for the proposed system}
\label{Chapter::Concepts for the proposed system}}\



\section{Background, objectives, and scope \label{Section::Background,objectives,and scope proposed}}
The fundamental rationale for the proposed ClauseGuard system is rooted in the complex and often unclear nature of contractual agreements. These agreements, crucial for a broad range of industries and transactions, are frequently laden with lengthy terms and conditions that may obscure potentially fraudulent or unfair clauses. It is an unfortunate reality that even conscientious individuals can overlook these elements due to the intricacy of legal language and the sheer volume of text.

This challenge is not limited to any specific domain but is rather a widespread issue across various sectors including, but not limited to, technology, insurance, financial services, real estate, medical, banking, insurance and retail. Each of these industries operates under the banner of contracts, making the identification and mitigation of dishonest clauses essential for ensuring fairness and transparency.

The proposed ClauseGuard system is designed to address this issue. Using advanced machine learning techniques, the system is envisioned to analyze and scrutinize contracts, accurately identifying and flagging any potentially fraudulent clauses. The system is not intended to merely detect these elements but also to provide a quantifiable risk assessment, thereby offering users a clear, actionable understanding of the contract's potential pitfalls. 

 The operational concept of ClauseGuard is tailored around a user-friendly, intuitive interface that facilitates easy upload and analysis of contracts. Users should be able to view the risk assessment and receive clear, understandable explanations for each identified risk, helping them make informed decisions.

The proposed system is expected to significantly enhance the fairness and transparency of contracts, leading to a more equitable business landscape. It will also serve as a valuable tool for individuals and businesses alike, empowering them to navigate contractual agreements with increased confidence and understanding.

The scope of ClauseGuard is ambitious, aiming to serve a broad range of industries. It is designed with a flexible architecture that can be tailored to meet specific industry needs, acknowledging the distinct language and terms used in different sectors.



\section{Operational policies and constraints \label{Section::Operational Policies and Constraints proposed}}
The operational policies and constraints of clauseguard are as follows: 
\begin{itemize}
    \item Operational Policies: 
    \begin{itemize}
        \item Data Privacy Policy: The ClauseGuard system is committed to handling all user data with the highest standards of confidentiality and privacy. User data will not be disclosed, sold, or shared with third parties without explicit user consent.

        \item Transparency Policy: The ClauseGuard system will provide clear explanations for each identified risk, ensuring that users understand the basis for each risk assessment. This includes transparency about the machine learning model's decision-making process to the extent possible without disclosing any information that may be used to unfairly target or harass the model or system itself. 
        \item Continuous Improvement Policy: The ClauseGuard system will regularly update its machine learning models based on user feedback and ongoing learning from new contracts to continuously improve its accuracy and reliability.
        
    \end{itemize}
    \item Operational Constraints: 
    \begin{itemize}
        \item System Availability: The ClauseGuard system is expected to operate 24/7, but there may be occasional downtime for system maintenance and upgrades. Notification of such downtime will be provided in advance when possible.
        \item Personnel Constraints: There may be limitations on the number of support staff available to assist users at any given time. This could lead to delayed response times during peak usage hours or during system outages.
        \item Hardware Constraints: The ClauseGuard system is designed to operate in a cloud-based environment, and as such, its performance may be affected by the performance and reliability of the cloud service provider.
        \item Hardware Constraints: The ClauseGuard system is designed to operate in a cloud-based environment, and as such, its performance may be affected by the performance and reliability of the cloud service provider.
    \end{itemize}
\end{itemize}

\section{Description of the proposed system \label{Section::Description of the proposed System proposed}}
The current system is being built with the primary objective of developing a machine learning model to aid stakeholders in identifying potentially deceptive clauses in terms and conditions (T\&C) contracts. This system functions by automatically scrutinizing all clauses within the contract, thereby equipping stakeholders with crucial information needed to make an informed decision regarding their agreement to the terms.
The system uses several machine learning libraries such as scikit-learn, natural language toolkit(NLTK) and keras for the creation of the model. 

Scikit-learn is a machine learning library in Python that provides tools for data analysis and modeling. In this system, scikit-learn is primarily used for preprocessing the dataset and training the machine learning model. It provides utilities for common machine learning tasks such as feature extraction, text representation, and model evaluation. For instance, scikit-learn's text feature extraction utilities can convert the text data in T&C contracts into a format that can be input to the machine learning model.
Natural Language Toolkit (NLTK) is a library in Python that provides tools for working with human language data (text).  In this system, NLTK is primarily used for tokenization (i.e. breaking up of sentences into each indidual alphabet, white space, special character, symbol etc), stemming (A process to remove the suffix from words such as ing), and lemmatization (converting the word into its root form and reducing the superlative of the word into its equivalent comparative, for example: oldest will be old), which are crucial steps in preprocessing the text data.

Keras is a high-level neural networks API in Python.  
It is used to define and train any kind of deep learning model. In this system, Keras is used to define the machine learning model structure and train the model using the preprocessed dataset.

Flask and Django are web development frameworks in Python.  In this system, Flask and Django are used to build the web application that allows users to upload T&C contracts and get the results from the machine learning model. Flask can handle simpler and smaller loads, while Django can manage more complex and larger loads. The choice between them would depend on the requirements and scalability plans of the system. Flask would be used for the inital build of the system followed by django for a more advanced model. 

The system operates by facilitating users to upload a T&C contract via a web interface, developed using Flask or Django. The uploaded contract undergoes a preprocessing step, where it's transformed into a suitable format for the machine learning model using libraries such as scikit-learn and NLTK.

The preprocessed data is then fed into a machine learning model, constructed and trained using the Keras library. This model evaluates each clause in the contract, assigning a risk score that represents the likelihood of the clause being deceptive or unfair.

The output provided to the user includes highlighted sections of the contract that have been flagged as potentially deceptive, along with a percentage indicator that quantifies the overall risk associated with the contract. For each highlighted section, the system also provides an explanation, derived from the model's analysis, clarifying why the clause was flagged.

With this information, the user is empowered to make an informed decision about whether to sign the contract or not. 

A more detailed analysis of the proposed system is given below: 
\begin{itemize}
    \item Operational Environment and Its Characteristics: The ClauseGuard system operates in a digital environment, primarily over a cloud-based platform. It is designed to be accessible to users worldwide, with the capacity to handle high volumes of concurrent user interactions.
    \item  Major System Components and Interconnections:
    \begin{itemize}
        \item User Interface: This is the primary point of interaction for users. It provides a simple and intuitive mechanism for users to upload contracts and view analysis results.
        \item Processing Engine: This is where the machine learning and natural language processing algorithms reside. It receives contract documents from the user interface, processes them, and sends the results back to the user interface.
        \item Data Storage: This component securely stores user data, system data, and analysis results. It interacts with both the user interface and the processing engine as needed.
    \end{itemize}

\item Interfaces to External Systems or Procedures:

The system is primarily standalone, but it may interface with external contract management systems or legal databases for contract reference purposes.

\item   Capabilities or Functions:

The primary function of the ClauseGuard system is to identify potentially fraudulent or misleading clauses in contracts using machine learning and natural language processing. The system will provide a risk percentage, highlight the fraudulent clauses, and offer explanations for its assessments.
\item Input, Output, Data Flow, and Processes:

Users input contract documents into the system. The processing engine analyzes these documents, identifies potential risks, and outputs these findings, including risk percentages and explanations, to the user interface. 

\item Cost of Systems Operations:

Operational costs will largely be driven by cloud hosting fees, ongoing system maintenance, and continuous model improvement. Specific cost details will be determined based on the chosen cloud provider and the scale of system usage.

\item Operational Risk Factors:

Risk factors include potential false positives or negatives in fraud detection, data security risks, and risks associated with system downtime or outages.

\item  Performance Characteristics:

The system is designed to quickly and accurately analyze contracts. Exact speed, throughput, volume, and frequency will be determined by system optimization and user demand.

\item  Quality Attributes:

The system prioritizes reliability, availability, correctness, efficiency, expandability, flexibility, maintainability, portability, reusability, and survivability.

\item Provisions for Safety, Security, Privacy, Integrity, and Continuity of Operations in Emergencies:

The system implements robust data security measures to ensure user data privacy and system integrity. In case of emergencies or outages, the system is designed for quick recovery, minimizing disruption to user services.


\end{itemize}



\section{Modes of operation \label{Section::Modes of operation proposed}}

The system will operate on various modes of operation throughout its lifecycle as detailed below: 
\begin{itemize}
    \item Training Mode: This mode is only accessible to the system's developers and not to the end users. It plays a crucial role in the functioning of the system as this is where new features are developed, the model is trained on additional datasets, and existing defects are patched. The training mode uses large amounts of T\&C contracts to train the machine learning model to understand the structure, language, and meaning of various clauses. This process allows the model to learn how to distinguish between benign and deceptive clauses. Enhancements and updates to the system are made in this mode before being deployed to the Operational Mode. 
    \item Operational Mode: This model is the primary end product of the system used by the users to upload T\&C contracts for analysis. The system would use the model developed in the training mode to evaluate the uploaded contract. It identifies potentially deceptive clauses, calculates the overall risk associated with the contract, and provides explanations for flagged sections. The results are then displayed to the user, who can use this information to make an informed decision about whether to sign the contract.
    \item Emergency Mode: This mode serves as a contingency plan, activated in the event that both the Primary and Training modes experience operational failures. The Emergency Mode relies on a legacy version of the currently deployed primary model. This fallback mechanism ensures continuity of service even in the face of unforeseen disruptions. While it may not have the most recent updates and features available in the Primary Mode, the Emergency Mode is designed to effectively analyze T&C contracts and identify deceptive clauses, thereby maintaining the core functionality of the system. Regular checks and minor updates are carried out to ensure that the Emergency Mode is always ready for activation, should the need arise. This robust backup plan contributes to the system's resilience and reliability, offering users a dependable tool for contract analysis at all times.
    


\end{itemize}


\section{User classes and other involved personnel \label{Section::User Classes and other involved personnel}}
Clauseguard would involve the participation several user classes and  personal either directly or indirectly depending upon their interaction with the system as given below: 
\begin{itemize}
    \item End User: The end users are the primary users of clauseguard. They interact directly with the system by uploading the T\&C contracts for analysis. Their interaction is via the web platform, with the aim of identifying deceptive contracts in T\&C contracts. They don't require any skill levels except basic computer skills. 
    \item Developers: They are the primary coders of the system. Their responsibilities include, implementing new features, functionality and maintenance of the system. They require high skill levels in the fields of programming and problem-solving as they require a high level of technical skill and familiarity with the underlying technology.
    \item Machine Learning Specialists: ML specialists work to continually improve clauseguard's machine learning algorithm. They are responsible for training the model on new datasets, testing performance and implementing updates as needed. Their interaction is through machine learning libraries used in the development of the model. 
    \item CyberSecurity Experts: These individuals play a crucial role in safeguarding the ClauseGuard system from unauthorized access and potential security threats and are arguably the most important stakeholders in the functioning of clauseguard as a service. They are tasked with the responsibility of thwarting attempts by unethical entities, often referred to as hackers, who might attempt to breach the system and gain access to confidential data. Cybersecurity experts work diligently to identify and secure any potential vulnerabilities in the system that could be exploited for malicious purposes. Their interaction with the system spans across all areas, but they primarily focus on the system's edge cases and potential weak points that could be susceptible to breaches. They are proficient in the latest cybersecurity protocols and techniques and utilize this expertise to constantly enhance the system's security measures. In addition, they collaborate closely with the system administrators and developers to ensure that security is integrated into all aspects of the ClauseGuard system, from the web application to the machine learning model itself. Their work helps maintain the trust of the end-users by ensuring that their interactions with the system remain secure and confidential.
    \item Legal Consultants: Although their involvement may not be direct, they nevertheless, play a crucial role in the development of the model. They review the machine learning model's output to ensure its accuracy and make recommendations for improvement based on their legal expertise. They interact indirectly with the system by providing feedback and suggestions for model enhancements based on their understanding of legal terminologies and concepts.
    \item Commercial enterprises stand to gain significantly from the use of the ClauseGuard system. These entities often engage in complex transactions that involve extensive legal contracts. With the help of ClauseGuard, they can quickly and efficiently review these contracts, expediting the deal-making process. For instance,  the  purchase of Activision-Blizzard by Microsoft made headlines across the tech industry. In such scenarios, missing or overlooking crucial legal details can lead to considerable impediments, potentially blocking the deal altogether which happened in the case of Microsoft acquiring Activision-Blizzard. ClauseGuard can mitigate such risks by identifying potentially harmful or unclear clauses that could jeopardize the agreement. This can save enterprises substantial time and resources, while also providing them with a layer of protection against contractual pitfalls. Similar to end users, commercial enterprises need only basic computing skills to access clauseguard. 

\end{itemize}


\section{User classes and other involved personnel \label{Section::User Classes and other involved personnel proposed}}
Clauseguard would involve the participation several user classes and  personal either directly or indrectly depending upon their interaction with the system as given below: 
\begin{itemize}
    \item End User: The end users are the primary users of clauseguard. They intract directly with the system by uploading the T\&C contracts for analysis. Their interaction is via the web platform, with the aim of identifying deceptive contracts in T\&C contracts. They don't rquire any skill levels except basic computer skills. 
    \item Developers: They are the primary coders of the system. Their responsibilities include, implementing new features, functionality and maintainnce of the system. They require high skill levels in the fields of programming and problem-solving as they require a high level of technical skill and familarity with the underlying technology.
    \item Machine Learning Specialists: ML specialists work to continully improve clauseguard's maachine learning algorithm. They are responsible for taining the model on new datasets, testing performance and implemting updates as needed. Their interaction is through machine leanring libraries used in the development of the model. 
    \item CyberSecurity Experts: These individuals play a crucial role in safeguarding the ClauseGuard system from unauthorized access and potential security threats and are arguably the most important stakeholders in the functioning of clauseguard as a service. They are tasked with the responsibility of thwarting attempts by unethical entities, often referred to as hackers, who might attempt to breach the system and gain access to confidential data. Cybersecurity experts work diligently to identify and secure any potential vulnerabilities in the system that could be exploited for malicious purposes. Their interaction with the system spans across all areas, but they primarily focus on the system's edge cases and potential weak points that could be susceptible to breaches. They are proficient in the latest cybersecurity protocols and techniques and utilize this expertise to constantly enhance the system's security measures. In addition, they collaborate closely with the system administrators and developers to ensure that security is integrated into all aspects of the ClauseGuard system, from the web application to the machine learning model itself. Their work helps maintain the trust of the end-users by ensuring that their interactions with the system remain secure and confidential.
    \item Legal Consultants: Although their invovlement may not be direct, they nevertheless, play a crucial role in the development of the model. They review the machine learning model's output to ensure its accuracy and make recommendations for improvement based on their legal expertise. They interact indirectly with the system by providing feedback and suggestions for model enhancements based on their understanding of legal terminologies and concepts.
    \item Commercial enterprises stand to gain significantly from the use of the ClauseGuard system. These entities often engage in complex transactions that involve extensive legal contracts. With the help of ClauseGuard, they can quickly and efficiently review these contracts, expediting the deal-making process. For instance,  the  purchase of Activision-Blizzard by Microsoft made headlines across the tech industry. In such scenarios, missing or overlooking crucial legal details can lead to considerable impediments, potentially blocking the deal altogether which happened in the case of Microsoft acquring Activision-Blizzard. ClauseGuard can mitigate such risks by identifying potentially harmful or unclear clauses that could jeopardize the agreement. This can save enterprises substantial time and resources, while also providing them with a layer of protection against contractual pitfalls. Similar to end users, commercial enterprieses need only basic computing skills to access clauseguard. 

\end{itemize}


\section{Support environment \label{Section::Support Environment proposed}}
User guides, online help, and a customer service team will all be part of the support system for the planned system. The system will also include a feedback tool that will let users comment on how well the system is working and recommend changes. To guarantee that the system is operating at its peak performance, routine upgrades and maintenance will be offered.

