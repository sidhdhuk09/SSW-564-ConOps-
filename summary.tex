\chapter{Summary of impacts \\
%\small{\textit{-- Team-11}} 
\index{Chapter!Summary of impacts }
\index{Summary of impacts}
\label{Chapter::Summary of impacts}}\
The proposed system "clauseguard" will have significant operational impacts on users, developers, and the support and maintenance organizations. 
For users, particularly legal and paralegal teams, the system seeks to induce change into the fundamentals of the existing workflow. Instead of manually reviewing contracts, users will now have the ability to upload them to the system and then analyze the generated reports to identify potentially fraudulent activities along with the explanations. This is expected to reduce the time spent on contract reviews, thereby increasing productivity and allowing more focus on complex cases that require human intervention.

Developers will face a shift in their roles as they will have to maintain and update the machine learning model regularly. This includes ensuring that the model is trained on up-to-date data, optimizing it for better performance, and resolving any issues that might arise.
Support and maintenance organizations will  need to familiarize themselves with the working of the ClauseGuard system. They will have to handle new kinds of queries and issues that may arise from the system's operation. The organizations will also have to develop a robust data backup and recovery system to protect the data fed into ClauseGuard.
It is to be noted that although these impacts are anticipated or expected, they might change over time during the actual implementation of the system.







\section{Operational impacts\label{Section::Operational Impacts}}
Operational impacts can be defined as how the system would evolve to become part of the new operational environment. The operational impacts are highlighted below: 
\begin{itemize}
    \item Since the model is platform independent, interfaces with primary or alternate computing platforms are not required except in the case of data recovery programs. 
    \item The procedure for reviewing contracts will change, as the model will be able to flag potentially deceptive clauses, speeding up the review process and reducing the burden on legal personnel.

    \item The model will require a database of contracts and legal clauses to function optimally. This data may be obtained from existing contract databases, open source legal databases, and other relevant sources.

    \item The system will now require digital versions of contracts to be input for analysis, which may increase the volume of data being handled. 
    \item Due to the need to train and improve the model, there may be changes in data retention policies, especially for contracts and related legal documents.

    \item Investment will be required for the development, implementation, and maintenance of the system, as well as additional cloud infrastructure required to effectively scale the system and ensure maximum availability to the tune of 99.999%  .
    \item While the system is expected to reduce the risk of entering into contracts with deceptive clauses, there may be new risks associated with data security and system reliability as well as the accuracy as existing laws are changed to support the ever evolving landscape of the legal world. 






\end{itemize}




\section{Organizational impacts \label{Section::Organizational Impacts}}
Organizational impact can be defined as the impact on the stakeholders of the system. The impact can be direct or indirect, positive or negative. The breakdown of organizational impact is given below: 
\begin{itemize}
    \item Legal personnel will need to review flagged clauses, in order to improve the accuracy of the model as their input determines the effective working of the model. 
    \item It personnel will have responsibilities for maintaining the system. 
    \item Positions related to manual contract review may be reduced, while positions related to system support and maintenance may increase as well as cloud infrastructure specialists and DevOps engineers. 
    \item Existing personnel will need to be trained to use the system and understand its outputs.

    \item There may be increased demand for personnel with skills in machine learning and contract law. Some personnel may need to be relocated to roles where these skills are more needed and where the model is being actively developed. 
    \item In case of an emergency or disaster that disrupts normal operations, contingency plans will need to be put in place. This will require a subset of personnel who are trained in operating the system to be available to set up and manage operations at an alternate site.

The number of personnel required will depend on the complexity of the system and the volume of contracts that need to be reviewed during the contingency operation. Given the technical nature of the system, these personnel should have a good understanding of machine learning principles, the specific model used, and the legal knowledge necessary to interpret the system's output.

Having a well-trained, technically competent contingency team will ensure that contract review can continue with minimal disruption, even under less than ideal circumstances. This is particularly important given the potential legal and financial implications of contracts with deceptive terms and conditions.




\end{itemize}




\section{Impacts during development\label{Section::Impacts During Development}} 
The impact of the system ClauseGuard during development is as follows detailed below: 
\begin{itemize}
    \item Users, developers, legal consultants, and support personnel will need to be involved in discussions and studies to define system requirements and objectives. The aforementioned stakeholders must be in consensus for the effective development of the system. Induced constraints by the way of conflicting requirements should be avoided at all costs to mitigate the risk of stalled development. 
    
    \item Users and support personnel will need to be involved in reviews and demonstrations of the system, to ensure it meets the organization's needs. They should receive additional training to ensure the maximum efficiency of the system. 

    \item During development, the new system may need to be run in parallel with existing procedures to ensure no disruption to contract review processes.



    \item System testing may require additional resources and may impact normal operations, as potential bugs and issues are identified and fixed.







\end{itemize}





